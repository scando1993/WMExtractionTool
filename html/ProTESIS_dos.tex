
% This LaTeX was auto-generated from an M-file by MATLAB.
% To make changes, update the M-file and republish this document.

\documentclass{article}
\usepackage{graphicx}
\usepackage{color}

\sloppy
\definecolor{lightgray}{gray}{0.5}
\setlength{\parindent}{0pt}

\begin{document}

    
    
\subsection*{Contents}

\begin{itemize}
\setlength{\itemsep}{-1ex}
   \item Dicom
   \item Descripcion
   \item C�digo
   \item elegir varias poblaciones
   \item create folder temporal
   \item delete folder temporal
   \item escojo los DICOM
\end{itemize}


\subsection*{Dicom}

\begin{par}
funcion que convierte las imagenes DICOM a formato .nii �til porque muchos esc�neres exportan sus datos en formato DICOM. Esta rutina convierte los ficheros DICOM en vol�menes de imagen compatibles con SPM
\end{par} \vspace{1em}
\begin{itemize}
\setlength{\itemsep}{-1ex}
   \item ESPOL     FIEC \& FIMCP    NBL"Neuroimaging \& Bioengineering Laboratory"
   \item Orlando Chancay
   \item $Id: organizarfoldersTESIS.m  23-Nov-2013  9:48:13z$
\end{itemize}


\subsection*{Descripcion}

\begin{par}
Esta funcion seleciona todas los cortes de una imagen y los convierte en unico volumen. se usa las funciones para selecionar ficheros como 'spm\_select', una funcion para obtener las cabezeras de las imagenes DICOM 'spm\_dicom\_headers', y la funcion 'spm\_dicom\_convert' que nos ayuda a convertir las imagenes a .nii
\end{par} \vspace{1em}


\subsection*{C�digo}

\begin{par}
la ruta de la carpeta de trabajo es enviada desde el programa principal y los nombre de los grupos. en las variable \textbf{ruta} , \textbf{nombresdepoblacion}
\end{par} \vspace{1em}


\subsection*{elegir varias poblaciones}



\subsection*{create folder temporal}



\subsection*{delete folder temporal}


\begin{verbatim}   for  i=1: length(subjects)
       [status,message,messageid] = rmdir(char(fullfile(matlabroot,poblacion(pobl),subjects{i},'TEMP6')),'s');
   end\end{verbatim}
    

\subsection*{escojo los DICOM}




\end{document}
    
