
% This LaTeX was auto-generated from an M-file by MATLAB.
% To make changes, update the M-file and republish this document.

\documentclass{article}
\usepackage{graphicx}
\usepackage{color}

\sloppy
\definecolor{lightgray}{gray}{0.5}
\setlength{\parindent}{0pt}

\begin{document}

    
    
\subsection*{Contents}

\begin{itemize}
\setlength{\itemsep}{-1ex}
   \item Organize
   \item Descripcion
   \item Código
   \item Base de datos de pacientes con las 2 imagenes
\end{itemize}


\subsection*{Organize}

\begin{par}
Funcion de ACL para organizar imagenes de RM contenidas en 'flair' a la Carpeta 'forg2' donde las imagenes estaran organizadas por paciente con sus respectivos nombres, tambien se comprueba que el paciente o control tenga las 2 imagenes T1 \& FLAIR, si no se eliminara automaticamente del estudio.
\end{par} \vspace{1em}
\begin{itemize}
\setlength{\itemsep}{-1ex}
   \item ESPOL     FIEC \& FIMCP    NBL"Neuroimaging \& Bioengineering Laboratory"
   \item Orlando Chancay
   \item $Id: organizarfoldersTESIS.m  23-Nov-2013  9:48:13z$
\end{itemize}


\subsection*{Descripcion}

\begin{par}
Crea la carpeta 'forg2' si no existe luego utiliza spm\_select para selecionar las carpetas de que contiene las imagenes DICOM y empieza a navegar dentro de las imagenes para ordenarlas y clasificarlas con el nombre del paciente. para obtener informacion acerca del paciente se utiliza la funcion \textbf{dicominfo} de la cual se puede obtener Nombre, edad, Tipo de imagen, Escaner y mucha informacion asociada con el pacienete y el Scaner
\end{par} \vspace{1em}


\subsection*{Código}

\begin{par}
la ruta de la carpeta de trabajo es enviada desde el programa principal en variable \textbf{ruta}
\end{par} \vspace{1em}


\subsection*{Base de datos de pacientes con las 2 imagenes}

\begin{par}
Almacena los nombres d elos pacientes en un archivo 'celldata2xx.txt' los cuales se los puede enlistar y nos serviran para continuar con el analisis en el estudio.
\end{par} \vspace{1em}



\end{document}
    
