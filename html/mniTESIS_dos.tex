
% This LaTeX was auto-generated from an M-file by MATLAB.
% To make changes, update the M-file and republish this document.

\documentclass{article}
\usepackage{graphicx}
\usepackage{color}

\sloppy
\definecolor{lightgray}{gray}{0.5}
\setlength{\parindent}{0pt}

\begin{document}

    
    
\subsection*{Contents}

\begin{itemize}
\setlength{\itemsep}{-1ex}
   \item MNI
   \item Descripci�n
   \item C�digo
   \item nuevo por archivo)
   \item nuevo por archivo)
\end{itemize}


\subsection*{MNI}

\begin{par}
Normalmente, Dartel genera im�genes deformadas que se alinean con la plantilla en forma de media \ensuremath{\backslash}cite\{ashburner2007fast\}. Esta rutina incluye un registro af�n inicial de la plantilla (el final uno generado por Dartel), con los datos TPM generados con SPM
\end{par} \vspace{1em}


\subsection*{Descripci�n}

\begin{par}
Esta rutina selecciona todas las im�genes SB\_Flair con \textbf{spm\_selec} luego toma los campos de deformacion generados por DARTEL \textbf{'\^{}u\_rc2ss.}\ensuremath{\backslash}.nii\$'*, y la Plantilla \textbf{'Template\_6.nii'} luego deforma las im�genes al espacio y forma de la plantilla.
\end{par} \vspace{1em}


\subsection*{C�digo}

\begin{par}
La funci�n necesita ruta de la carpeta de trabajo la cual debe ser enviada desde el programa principal y los nombre de los grupos que creamos, en las variable \textbf{ruta} , \textbf{nombresdepoblacion} .
\end{par} \vspace{1em}
\begin{par}
matlabroot='/Users/orlando/Documents/MATLAB/TESIS/test';
\end{par} \vspace{1em}


\subsection*{nuevo por archivo)}



\subsection*{nuevo por archivo)}




\end{document}
    
