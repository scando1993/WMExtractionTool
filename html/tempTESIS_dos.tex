
% This LaTeX was auto-generated from an M-file by MATLAB.
% To make changes, update the M-file and republish this document.

\documentclass{article}
\usepackage{graphicx}
\usepackage{color}

\sloppy
\definecolor{lightgray}{gray}{0.5}
\setlength{\parindent}{0pt}

\begin{document}

    
    
\subsection*{Contents}

\begin{itemize}
\setlength{\itemsep}{-1ex}
   \item Template
   \item Descripci�n
   \item C�digo
   \item elegir varias poblaciones
\end{itemize}


\subsection*{Template}

\begin{par}
Funci�n que sirve para crear una plantilla a partir de todas las im�genes con la ayuda de \textbf{DARTEL}  'A Fast Diffeomorphic Registration Algorithm' \ensuremath{\backslash}cite\{ashburner2007fast\} . de los sujetos del estudio. en nuestra tesis solamente creamos una plantilla con la Sustancia Blanca. La cual nos sirva para normalizar las im�genes y poder hacer comparaciones y �an�lisis de ANOVA.
\end{par} \vspace{1em}


\subsection*{Descripci�n}

\begin{par}
Esta funci�n selecciona todas las im�genes SB\_T1 con \textbf{spm\_selec} de la imagen T1 y crea nuestra plantilla mediante la funci�n de SPM \textbf{dartel} .
\end{par} \vspace{1em}


\subsection*{C�digo}

\begin{par}
La funci�n necesita ruta de la carpeta de trabajo la cual debe ser enviada desde el programa principal y los nombre de los grupos que creamos.
\end{par} \vspace{1em}
\begin{par}
matlabroot='/Users/orlando/Documents/MATLAB/TESIS/test';
\end{par} \vspace{1em}


\subsection*{elegir varias poblaciones}




\end{document}
    
