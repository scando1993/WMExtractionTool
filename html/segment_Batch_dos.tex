
% This LaTeX was auto-generated from an M-file by MATLAB.
% To make changes, update the M-file and republish this document.

\documentclass{article}
\usepackage{graphicx}
\usepackage{color}

\sloppy
\definecolor{lightgray}{gray}{0.5}
\setlength{\parindent}{0pt}

\begin{document}

    
    
\subsection*{Contents}

\begin{itemize}
\setlength{\itemsep}{-1ex}
   \item Segment
   \item Descripcion
   \item C�digo
\end{itemize}


\subsection*{Segment}

\begin{par}
Funcion que Segmenta las imagenes T1 y aplica una correci�n de Bias. En SPM se utiliza una versi�n modificada del algoritmo mixture model. Se asume que las im�genes de RM consisten de un n�mero de distintos tipos de tejido (clusters) a partir de los cuales cada v�xel ha sido dibujado
\end{par} \vspace{1em}


\subsection*{Descripcion}

\begin{par}
Esta funcion seleciona todas las imagenes T1 con \textbf{spm\_selec} aplica un smooth al histograma con la funcion \textbf{spm\_smooth} usamos mapas de probabilidades de los tejidos que los proporciona SPM. ademas para nuestro trabajo debemos setear par�metros para que la funcion nos arroje como resultado los tejidos en en espacio nativo y en el  espacio MNI para que estas puedan ser utilizadas por \textbf{DARTEL} al moneto de crear la plantilla.
\end{par} \vspace{1em}


\subsection*{C�digo}

\begin{par}
La funcion necesita ruta de la carpeta de trabajo la cual debe ser enviada desde el programa principal y los nombre de los grupos que creamos. en las variable \textbf{ruta} , \textbf{nombresdepoblacion}
\end{par} \vspace{1em}



\end{document}
    
