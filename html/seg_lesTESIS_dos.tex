
% This LaTeX was auto-generated from an M-file by MATLAB.
% To make changes, update the M-file and republish this document.

\documentclass{article}
\usepackage{graphicx}
\usepackage{color}

\sloppy
\definecolor{lightgray}{gray}{0.5}
\setlength{\parindent}{0pt}

\begin{document}

    
    
\subsection*{Contents}

\begin{itemize}
\setlength{\itemsep}{-1ex}
   \item Lesions NE \& MNI
   \item Descripci�n
   \item C�digo
   \item Example to read a volume (an .img image)
   \item elegir varias poblaciones
   \item imrimir lesiones en vlumen
\end{itemize}


\subsection*{Lesions NE \& MNI}

\begin{par}
Las Lesiones se presentan en las im�genes FLAIR como hiperintensidades, la segmentacion de lesiones se basa en localizar un treshold dinamico. El cual sera un discriminador de voxeles.
\end{par} \vspace{1em}


\subsection*{Descripci�n}

\begin{par}
Esta rutina permite segementar lesiones en sustancia blanca, selecciona todas las im�genes SB\_FLAIR con \textbf{spm\_selec} luego mediante un lazo segmenta las lesiones de cada una de ellas.
\end{par} \vspace{1em}


\subsection*{C�digo}

\begin{par}
La funci�n necesita la ruta de la carpeta de trabajo la cual debe ser enviada desde el programa principal, los nombre de los grupos que creamos y el espacio(nativo y mni) en las variables. \textbf{(ruta,espacio,nombresdepoblacion)}
\end{par} \vspace{1em}


\subsection*{Example to read a volume (an .img image)}

\begin{par}
clc
\end{par} \vspace{1em}


\subsection*{elegir varias poblaciones}

\begin{par}
if nVarargs == 0;
\end{par} \vspace{1em}


\subsection*{imrimir lesiones en vlumen}




\end{document}
    
