
% This LaTeX was auto-generated from an M-file by MATLAB.
% To make changes, update the M-file and republish this document.

\documentclass{article}
\usepackage{graphicx}
\usepackage{color}

\sloppy
\definecolor{lightgray}{gray}{0.5}
\setlength{\parindent}{0pt}

\begin{document}

    
    
\subsection*{Contents}

\begin{itemize}
\setlength{\itemsep}{-1ex}
   \item Classify
   \item Descripcion
   \item C�digo
\end{itemize}


\subsection*{Classify}

\begin{par}
Funcion de ACL para clasificar las imagenes en diferentes patologias en nuestro caso tenemos que crear 3 Grupos 'HC' controles sanos, 'LPD' Parkinsons Izquierdos, 'RPD' Parkinsons Derechos. Al utilizar esta opcion tenemos que ingresar en numero y los nombres de los grupos que queremos crear. Tambien tenemos que selecionar en una tabla mediante un check a que grupo queremos que pertenesca nuestras im�genes.
\end{par} \vspace{1em}
\begin{itemize}
\setlength{\itemsep}{-1ex}
   \item ESPOL     FIEC \& FIMCP    NBL"Neuroimaging \& Bioengineering Laboratory"
   \item Orlando Chancay
   \item $Id: organizarfoldersTESIS.m  23-Nov-2013  9:48:13z$
\end{itemize}


\subsection*{Descripcion}

\begin{par}
Crea carpetas las que el usuario selecion en nuestro caso crearemos 3 carpetas 'HC', 'LPD', 'RPD. Se crea una tabla con los nombres de los sujetos en la cual tenemos que selecionar a que Grupo de paciente queremos que pertenesca, una vez selecionado y presionado el boton Save\&OK las imagenes son desplazadas a las carpetas selecionadas. se usa las funciones 'spm\_select', funciones de archivo como 'fopen' y funciones de tablas 'uitable'.
\end{par} \vspace{1em}


\subsection*{C�digo}

\begin{par}
la ruta de la carpeta de trabajo es enviada desde el programa principal como la tabla y los nombre de los grupos. en las variable \textbf{ruta}  , \textbf{tabla}, \textbf{nombredegrupos}
\end{par} \vspace{1em}



\end{document}
    
