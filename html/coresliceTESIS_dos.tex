
% This LaTeX was auto-generated from an M-file by MATLAB.
% To make changes, update the M-file and republish this document.

\documentclass{article}
\usepackage{graphicx}
\usepackage{color}

\sloppy
\definecolor{lightgray}{gray}{0.5}
\setlength{\parindent}{0pt}

\begin{document}

    
    
\subsection*{Contents}

\begin{itemize}
\setlength{\itemsep}{-1ex}
   \item Coreslice
   \item Descripcion
   \item C�digo
\end{itemize}


\subsection*{Coreslice}

\begin{par}
funcion que Coregistra las imagenes T1 y FLAIR y le aplica una interpolacion a las imagen FLIAR que posee 36 cortes.
\end{par} \vspace{1em}
\begin{itemize}
\setlength{\itemsep}{-1ex}
   \item ESPOL     FIEC \& FIMCP    NBL"Neuroimaging \& Bioengineering Laboratory"
   \item Orlando Chancay
   \item $Id: organizarfoldersTESIS.m  23-Nov-2013  9:48:13z$
\end{itemize}


\subsection*{Descripcion}

\begin{par}
Esta funcion seleciona todas las imagenes T1 y FLAIR y las coregistra mediante la funcion 'spm\_coreg', para leer las caebzaras de las iamgenes usamos la funcion 'spm\_vol', obtenemos el espacio de la T1 mediante la funcion spm\_get\_space y por �ltimo para interpolas cortes intermedios de la imagen FLAIR de 36 cortes se usa la funcion spm\_reslice
\end{par} \vspace{1em}


\subsection*{C�digo}

\begin{par}
la ruta de la carpeta de trabajo es enviada desde el programa principal y los nombre de los grupos. en las variable \textbf{ruta} , \textbf{nombresdepoblacion} Opciones seteadas:         defaults.coreg.estimate.cost\_fun = 'nmi';         defaults.coreg.estimate.sep      = [4 2];         defaults.coreg.estimate.tol      = [0.02 0.02 0.02 0.001 0.001 0.001 0.01 0.01 0.01 0.001 0.001 0.001];         defaults.coreg.estimate.fwhm     = [7 7];
\end{par} \vspace{1em}



\end{document}
    
